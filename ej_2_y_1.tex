\documentclass[50pt,a4paper]{article}

\input{AEDmacros}
\usepackage{caratula}
\usepackage{mathtools}

\graphicspath{{../static/}}
\begin{document}

\section{Especificación}


\section{Demostraciones de correctitud}
\subsection{Introducción}

Para demostrar la correctitud del programa en base a la especificación, se utilizará la lógica de la tripla de Hoare:
\begin{align*}
    \{P\} S \{Q\}
    \shortintertext{\parbox{\textwidth}{\centering \textit {\footnotesize {Donde P es la precondición del programa, S el programa y Q la postcondición.}}}}
\end{align*}
\indent
En esta lógica, S es correcto para P y Q si y sólo si P implica la precondicion más débil de S para Q.
Es decir:
\begin{align*}
    \{P\} S \{Q\} \iff P \implica WP(S, Q)
\end{align*}
\indent
Como el programa contiene un ciclo, para simplificar la demostración, se dividirá en tres partes.
En primer lugar, se comprobará que: 
\begin{align}
    P \implica WP(S_{1}, Pc)
    \label{eq:1}
    \shortintertext{\parbox{\textwidth}{\centering \textit {\footnotesize {Donde P es la precondición del programa, $S_{1}$ la parte del programa previa al ciclo y Pc la precondición del ciclo.}}}} \notag
\end{align}
\indent
Luego:
\begin{align}
    Pc \implica WP(Sc, Qc)
    \label{eq:2}
    \shortintertext{\parbox{\textwidth}{\centering \textit {\footnotesize {Donde Pc es la precondición del ciclo, Sc la parte del programa a ejecutar dentro del ciclo y Qc la postcondición del ciclo.}}}} \notag
\end{align}
\indent
Y finalmente:
\begin{align}
    Qc \implica WP(S_{2}, Q)
    \label{eq:3}
    \shortintertext{\parbox{\textwidth}{\centering \textit {\footnotesize {Donde Qc es la postcondición del ciclo, $S_{2}$ la parte del programa posterior al ciclo y Q la postcondición del programa.}}}} \notag
\end{align}
\indent
Se definen las variables P, Q, Pc, Qc, I, B y la función Fv, que serán utilizadas a lo largo de la demostración:
\begin{align*}
    P &= \{ \existe[unalinea]{j}{\ent}{0 \leq i < |ciudades| \yLuego ciudades[j]}\\
    &\land \paraTodo[unalinea]{j}{\ent}{0 \leq j < |ciudades| \implicaLuego ciudades[j].habitantes \geq 0}\\
    &\land \paraTodo[unalinea]{k, h}{\ent}{0 \leq k < h < |ciudades| \implicaLuego ciudades[k].nombre \neq ciudades[h].nombre} \}\\
    Q &= \{ res = \sum\limits_{j=0}^{|ciudades| - 1} ciudades[j].habitantes \land i = |ciudades| \}\\
    Pc &= \{ res = 0 \land i = 0 \}\\
    Qc &= \{ res = \sum\limits_{j=0}^{|ciudades| - 1} ciudades[j].habitantes \land i = |ciudades| \}\\
    I &= \{ 0 \leq i \leq |ciudades| \yLuego res = \sum\limits_{j=0}^{i - 1} ciudades[j].habitantes \}\\
    B &= \{ i < |ciudades| \}\\
    Fv(i) &= |ciudades| - i
\end{align*}
\newpage
\subsection{Demostración}

Comenzamos probando (1):
\begin{align*}
    P &\implica WP(res:=0;i:=0, Pc)\\
    &\qquad\qquad\iff \text{por axioma 3}\\
    P &\implica WP(res:=0, WP(i:=0, Pc))\\
    &\qquad\qquad\iff\\
    P &\implica WP(res:=0, WP(i:=0, \{ res = 0 \land i = 0 \}))\\
    &\qquad\qquad\iff \text{por axioma 1}\\
    P &\implica WP(res:=0, \{ def(0) \yLuego  Pc_0^i \})\\
    &\qquad\qquad\iff\\
    P &\implica WP(res:=0, \{ def(0) \yLuego  res = 0 \land 0 = 0 \})\\
    &\qquad\qquad\iff\\
    P &\implica WP(res:=0, \{ true \yLuego res = 0 \land 0 = 0 \})\\
    &\qquad\qquad\iff\\
    P &\implica WP(res:=0, \{ res = 0 \land true \})\\
    &\qquad\qquad\iff\\
    P &\implica WP(res:=0, \{ res = 0 \})\\
    q &= \{ res = 0 \}\\
    &\qquad\qquad\iff \text{por axioma 1}\\
    P &\implica WP(res:=0, \{ def(0) \yLuego  q_0^{res} \})\\
    &\qquad\qquad\iff\\
    P &\implica \{ true \land 0 = 0 \}\\
    &\qquad\qquad\iff\\
    P &\implica \{ true \}\\
    \shortintertext{\parbox{\textwidth}{\centering \small {Que es tautología, por lo que queda probado.}}} \notag
\end{align*}
\indent
Continuamos con (3):
\begin{align*}
    Qc &\implica Q
    \shortintertext{\parbox{\textwidth}{\centering \small {Y esto vale porque Qc es igual a Q}}} \notag
\end{align*}
\indent
Finalmente (2) se demostrará con la correctitud del ciclo, primero con el teorema del invariante y luego con el teorema de terminación del ciclo.\\
\indent
Teorema del invariante:\\
\begin{align*}
    &a.\quad Pc \implica I\\
    &b.\quad \{ I \land B \} \quad S \quad \{ I \}\\
    &c.\quad I \land \neg B \implica Qc\\
\end{align*}
\indent
La proposición (a), se prueba de la siguiente manera:
\begin{align*}  % pensar si cambiamos los equiv por iff
    &\{ res = 0 \land i = 0 \} \implica \{ 0 \leq i \leq |ciudades| \yLuego res = \sum\limits_{j=0}^{i - 1} ciudades[j].habitantes \} \equiv \\
    \shortintertext{\parbox{\textwidth}{\centering \small {Se toma Pc verdadero, entonces res = 0 y i = 0 valen para I}}} \notag
    &\equiv\{ res = 0 \land i = 0 \} \implica \{ 0 = 0 \leq |ciudades| \yLuego 0 = \sum\limits_{j=0}^{0 - 1} ciudades[j].habitantes \} \equiv\\
    \shortintertext{\parbox{\textwidth}{\centering \small {El rango negativo para la sumatoria la reduce a 0}}} \notag
    &\equiv\{ res = 0 \land i = 0 \} \implica \{ 0 = 0 \leq |ciudades| \yLuego 0 = 0 \} \equiv\\
    &\equiv\{ res = 0 \land i = 0 \} \implica \{ 0 \leq |ciudades| \yLuego true \} \equiv\\
    &\equiv\{ res = 0 \land i = 0 \} \implica \{ 0 \leq |ciudades| \} \equiv\\
    \shortintertext{\parbox{\textwidth}{\centering \small {Como el módulo como mínimo es 0, esto también vale}}} \notag
    &\equiv\{ res = 0 \land i = 0 \} \implica \{ true \} \\
    \shortintertext{\parbox{\textwidth}{\centering \small {Y finalmente esta expresión es tautología}}} \notag
\end{align*}
\indent
La proposición (b), se prueba de la siguiente manera:
\begin{align*}
    \shortintertext{\parbox{\textwidth}{\centering \small {La tripla de Hoare es verdadera si y sólo si}}} \notag
    \{ 0 \leq i \leq |ciudades| \yLuego res &= \sum\limits_{j=0}^{i - 1} ciudades[j].habitantes \land i < |ciudades| \}\\
     &\implica \\
     WP(Sc, 0 \leq i \leq |ciudades| \yLuego res &= \sum\limits_{j=0}^{i - 1} ciudades[j].habitantes)\\
     &\iff\\
     
\end{align*}
\indent
En conclusión, por monotonía, al haber probado (1), (2) y (3) queda demostrada la correctitud del programa según la lógica de la tripla de Hoare.

\end{document}
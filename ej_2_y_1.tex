\documentclass[50pt,a4paper]{article}

\usepackage[spanish,activeacute,es-tabla]{babel}
\usepackage[utf8]{inputenc}
\usepackage{ifthen}
\usepackage{listings}
\usepackage{dsfont}
\usepackage{subcaption}
\usepackage{amsmath}
\usepackage[strict]{changepage}
\usepackage[top=1cm,bottom=2cm,left=1cm,right=1cm]{geometry}%
\usepackage{color}%
\newcommand{\tocarEspacios}{%
	\addtolength{\leftskip}{3em}%
	\setlength{\parindent}{0em}%
}

% Especificacion de procs

\newcommand{\In}{\textsf{in }}
\newcommand{\Out}{\textsf{out }}
\newcommand{\Inout}{\textsf{inout }}

\newcommand{\encabezadoDeProc}[4]{%
	% Ponemos la palabrita problema en tt
	%  \noindent%
	{\normalfont\bfseries\ttfamily proc}%
	% Ponemos el nombre del problema
	\ %
	{\normalfont\ttfamily #2}%
	\
	% Ponemos los parametros
	(#3)%
	\ifthenelse{\equal{#4}{}}{}{%
		% Por ultimo, va el tipo del resultado
		\ : #4}
}

\newenvironment{proc}[4][res]{%
	
	% El parametro 1 (opcional) es el nombre del resultado
	% El parametro 2 es el nombre del problema
	% El parametro 3 son los parametros
	% El parametro 4 es el tipo del resultado
	% Preambulo del ambiente problema
	% Tenemos que definir los comandos requiere, asegura, modifica y aux
	\newcommand{\requiere}[2][]{%
		{\normalfont\bfseries\ttfamily requiere}%
		\ifthenelse{\equal{##1}{}}{}{\ {\normalfont\ttfamily ##1} :}\ %
		\{\ensuremath{##2}\}%
		{\normalfont\bfseries\,\par}%
	}
	\newcommand{\asegura}[2][]{%
		{\normalfont\bfseries\ttfamily asegura}%
		\ifthenelse{\equal{##1}{}}{}{\ {\normalfont\ttfamily ##1} :}\
		\{\ensuremath{##2}\}%
		{\normalfont\bfseries\,\par}%
	}
	\renewcommand{\aux}[4]{%
		{\normalfont\bfseries\ttfamily aux\ }%
		{\normalfont\ttfamily ##1}%
		\ifthenelse{\equal{##2}{}}{}{\ (##2)}\ : ##3\, = \ensuremath{##4}%
		{\normalfont\bfseries\,;\par}%
	}
	\renewcommand{\pred}[3]{%
		{\normalfont\bfseries\ttfamily pred }%
		{\normalfont\ttfamily ##1}%
		\ifthenelse{\equal{##2}{}}{}{\ (##2) }%
		\{%
		\begin{adjustwidth}{+5em}{}
			\ensuremath{##3}
		\end{adjustwidth}
		\}%
		{\normalfont\bfseries\,\par}%
	}
	
	\newcommand{\res}{#1}
	\vspace{1ex}
	\noindent
	\encabezadoDeProc{#1}{#2}{#3}{#4}
	% Abrimos la llave
	\par%
	\tocarEspacios
}
{
	% Cerramos la llave
	\vspace{1ex}
}

\newcommand{\aux}[4]{%
	{\normalfont\bfseries\ttfamily\noindent aux\ }%
	{\normalfont\ttfamily #1}%
	\ifthenelse{\equal{#2}{}}{}{\ (#2)}\ : #3\, = \ensuremath{#4}%
	{\normalfont\bfseries\,;\par}%
}

\newcommand{\pred}[3]{%
	{\normalfont\bfseries\ttfamily\noindent pred }%
	{\normalfont\ttfamily #1}%
	\ifthenelse{\equal{#2}{}}{}{\ (#2) }%
	\{%
	\begin{adjustwidth}{+2em}{}
		\ensuremath{#3}
	\end{adjustwidth}
	\}%
	{\normalfont\bfseries\,\par}%
}

% Tipos

\newcommand{\nat}{\ensuremath{\mathds{N}}}
\newcommand{\ent}{\ensuremath{\mathds{Z}}}
\newcommand{\float}{\ensuremath{\mathds{R}}}
\newcommand{\bool}{\ensuremath{\mathsf{Bool}}}
\newcommand{\cha}{\ensuremath{\mathsf{Char}}}
\newcommand{\str}{\ensuremath{\mathsf{String}}}

% Logica

\newcommand{\True}{\ensuremath{\mathrm{true}}}
\newcommand{\False}{\ensuremath{\mathrm{false}}}
\newcommand{\Then}{\ensuremath{\rightarrow}}
\newcommand{\Iff}{\ensuremath{\leftrightarrow}}
\newcommand{\implica}{\ensuremath{\longrightarrow}}
\newcommand{\IfThenElse}[3]{\ensuremath{\mathsf{if}\ #1\ \mathsf{then}\ #2\ \mathsf{else}\ #3\ \mathsf{fi}}}
\newcommand{\yLuego}{\land _L}
\newcommand{\oLuego}{\lor _L}
\newcommand{\implicaLuego}{\implica _L}

\newcommand{\cuantificador}[5]{%
	\ensuremath{(#2 #3: #4)\ (%
		\ifthenelse{\equal{#1}{unalinea}}{
			#5
		}{
			$ % exiting math mode
			\begin{adjustwidth}{+2em}{}
				$#5$%
			\end{adjustwidth}%
			$ % entering math mode
		}
		)}
}

\newcommand{\existe}[4][]{%
	\cuantificador{#1}{\exists}{#2}{#3}{#4}
}
\newcommand{\paraTodo}[4][]{%
	\cuantificador{#1}{\forall}{#2}{#3}{#4}
}

%listas

\newcommand{\TLista}[1]{\ensuremath{seq \langle #1\rangle}}
\newcommand{\lvacia}{\ensuremath{[\ ]}}
\newcommand{\lv}{\ensuremath{[\ ]}}
\newcommand{\longitud}[1]{\ensuremath{|#1|}}
\newcommand{\cons}[1]{\ensuremath{\mathsf{addFirst}}(#1)}
\newcommand{\indice}[1]{\ensuremath{\mathsf{indice}}(#1)}
\newcommand{\conc}[1]{\ensuremath{\mathsf{concat}}(#1)}
\newcommand{\cab}[1]{\ensuremath{\mathsf{head}}(#1)}
\newcommand{\cola}[1]{\ensuremath{\mathsf{tail}}(#1)}
\newcommand{\sub}[1]{\ensuremath{\mathsf{subseq}}(#1)}
\newcommand{\en}[1]{\ensuremath{\mathsf{en}}(#1)}
\newcommand{\cuenta}[2]{\mathsf{cuenta}\ensuremath{(#1, #2)}}
\newcommand{\suma}[1]{\mathsf{suma}(#1)}
\newcommand{\twodots}{\ensuremath{\mathrm{..}}}
\newcommand{\masmas}{\ensuremath{++}}
\newcommand{\matriz}[1]{\TLista{\TLista{#1}}}
\newcommand{\seqchar}{\TLista{\cha}}
\newcommand{\listavacia}{\lange \rangle}

\renewcommand{\lstlistingname}{Código}
\lstset{% general command to set parameter(s)
	language=Java,
	morekeywords={endif, endwhile, skip},
	basewidth={0.47em,0.40em},
	columns=fixed, fontadjust, resetmargins, xrightmargin=5pt, xleftmargin=15pt,
	flexiblecolumns=false, tabsize=4, breaklines, breakatwhitespace=false, extendedchars=true,
	numbers=left, numberstyle=\tiny, stepnumber=1, numbersep=9pt,
	frame=l, framesep=3pt,
	captionpos=b,
}

\usepackage{caratula}
\usepackage{mathtools}

\graphicspath{{../static/}}
\begin{document}

\section{Especificación}


\section{Demostraciones de correctitud}
\subsection{Introducción}

Para demostrar la correctitud del programa en base a la especificación, se utilizará la lógica de la tripla de Hoare:
\begin{align*}
    \{P\} S \{Q\}
    \shortintertext{\parbox{\textwidth}{\centering \textit {\footnotesize {Donde P es la precondición del programa, S el programa y Q la postcondición.}}}}
\end{align*}
\indent
En esta lógica, S es correcto para P y Q si y sólo si P implica la precondicion más débil de S para Q.
Es decir:
\begin{align*}
    \{P\} S \{Q\} \iff P \implica WP(S, Q)
\end{align*}
\indent
Como el programa contiene un ciclo, para simplificar la demostración, se dividirá en tres partes.
En primer lugar, se comprobará que: 
\begin{align}
    P \implica WP(S_{1}, Pc)
    \label{eq:1}
    \shortintertext{\parbox{\textwidth}{\centering \textit {\footnotesize {Donde P es la precondición del programa, $S_{1}$ la parte del programa previa al ciclo y Pc la precondición del ciclo.}}}} \notag
\end{align}
\indent
Luego:
\begin{align}
    Pc \implica WP(Sc, Qc)
    \label{eq:2}
    \shortintertext{\parbox{\textwidth}{\centering \textit {\footnotesize {Donde Pc es la precondición del ciclo, Sc la parte del programa a ejecutar dentro del ciclo y Qc la postcondición del ciclo.}}}} \notag
\end{align}
\indent
Y finalmente:
\begin{align}
    Qc \implica WP(S_{2}, Q)
    \label{eq:3}
    \shortintertext{\parbox{\textwidth}{\centering \textit {\footnotesize {Donde Qc es la postcondición del ciclo, $S_{2}$ la parte del programa posterior al ciclo y Q la postcondición del programa.}}}} \notag
\end{align}
\indent
Se definen las variables P, Q, Pc, Qc, I, B y la función Fv, que serán utilizadas a lo largo de la demostración:
\begin{align*}
    P &= \{ \existe[unalinea]{j}{\ent}{0 \leq i < |ciudades| \yLuego ciudades[j]}\\
    &\land \paraTodo[unalinea]{j}{\ent}{0 \leq j < |ciudades| \implicaLuego ciudades[j].habitantes \geq 0}\\
    &\land \paraTodo[unalinea]{k, h}{\ent}{0 \leq k < h < |ciudades| \implicaLuego ciudades[k].nombre \neq ciudades[h].nombre} \}\\
    Q &= \{ res = \sum\limits_{j=0}^{|ciudades| - 1} ciudades[j].habitantes \land i = |ciudades| \}\\
    Pc &= \{ res = 0 \land i = 0 \}\\
    Qc &= \{ res = \sum\limits_{j=0}^{|ciudades| - 1} ciudades[j].habitantes \land i = |ciudades| \}\\
    I &= \{ 0 \leq i \leq |ciudades| \yLuego res = \sum\limits_{j=0}^{i - 1} ciudades[j].habitantes \}\\
    B &= \{ i < |ciudades| \}\\
    Fv(i) &= |ciudades| - i
\end{align*}
\newpage
\subsection{Demostración}

Comenzamos probando (1):
\begin{align*}
    P &\implica WP(res:=0;i:=0, Pc)\\
    &\qquad\qquad\iff \text{por axioma 3}\\
    P &\implica WP(res:=0, WP(i:=0, Pc))\\
    &\qquad\qquad\iff\\
    P &\implica WP(res:=0, WP(i:=0, \{ res = 0 \land i = 0 \}))\\
    &\qquad\qquad\iff \text{por axioma 1}\\
    P &\implica WP(res:=0, \{ def(0) \yLuego  Pc_0^i \})\\
    &\qquad\qquad\iff\\
    P &\implica WP(res:=0, \{ def(0) \yLuego  res = 0 \land 0 = 0 \})\\
    &\qquad\qquad\iff\\
    P &\implica WP(res:=0, \{ true \yLuego res = 0 \land 0 = 0 \})\\
    &\qquad\qquad\iff\\
    P &\implica WP(res:=0, \{ res = 0 \land true \})\\
    &\qquad\qquad\iff\\
    P &\implica WP(res:=0, \{ res = 0 \})\\
    q &= \{ res = 0 \}\\
    &\qquad\qquad\iff \text{por axioma 1}\\
    P &\implica WP(res:=0, \{ def(0) \yLuego  q_0^{res} \})\\
    &\qquad\qquad\iff\\
    P &\implica \{ true \land 0 = 0 \}\\
    &\qquad\qquad\iff\\
    P &\implica \{ true \}\\
    \shortintertext{\parbox{\textwidth}{\centering \small {Que es tautología, por lo que queda probado.}}} \notag
\end{align*}
\indent
Continuamos con (3):
\begin{align*}
    Qc &\implica Q
    \shortintertext{\parbox{\textwidth}{\centering \small {Y esto vale porque Qc es igual a Q}}} \notag
\end{align*}
\indent
Finalmente (2) se demostrará con la correctitud del ciclo, primero con el teorema del invariante y luego con el teorema de terminación del ciclo.\\
\indent
Teorema del invariante:\\
\begin{align*}
    &a.\quad Pc \implica I\\
    &b.\quad \{ I \land B \} \quad S \quad \{ I \}\\
    &c.\quad I \land \neg B \implica Qc\\
\end{align*}
\indent
La proposición (a), se prueba de la siguiente manera:
\begin{align*}  % pensar si cambiamos los equiv por iff
    &\{ res = 0 \land i = 0 \} \implica \{ 0 \leq i \leq |ciudades| \yLuego res = \sum\limits_{j=0}^{i - 1} ciudades[j].habitantes \} \equiv \\
    \shortintertext{\parbox{\textwidth}{\centering \small {Se toma Pc verdadero, entonces res = 0 y i = 0 valen para I}}} \notag
    &\equiv\{ res = 0 \land i = 0 \} \implica \{ 0 = 0 \leq |ciudades| \yLuego 0 = \sum\limits_{j=0}^{0 - 1} ciudades[j].habitantes \} \equiv\\
    \shortintertext{\parbox{\textwidth}{\centering \small {El rango negativo para la sumatoria la reduce a 0}}} \notag
    &\equiv\{ res = 0 \land i = 0 \} \implica \{ 0 = 0 \leq |ciudades| \yLuego 0 = 0 \} \equiv\\
    &\equiv\{ res = 0 \land i = 0 \} \implica \{ 0 \leq |ciudades| \yLuego true \} \equiv\\
    &\equiv\{ res = 0 \land i = 0 \} \implica \{ 0 \leq |ciudades| \} \equiv\\
    \shortintertext{\parbox{\textwidth}{\centering \small {Como el módulo como mínimo es 0, esto también vale}}} \notag
    &\equiv\{ res = 0 \land i = 0 \} \implica \{ true \} \\
    \shortintertext{\parbox{\textwidth}{\centering \small {Y finalmente esta expresión es tautología}}} \notag
\end{align*}
\indent
La proposición (b), se prueba de la siguiente manera:
\begin{align*}
    \shortintertext{\parbox{\textwidth}{\centering \small {La tripla de Hoare es verdadera si y sólo si}}} \notag
    \{ 0 \leq i \leq |ciudades| \yLuego res &= \sum\limits_{j=0}^{i - 1} ciudades[j].habitantes \land i < |ciudades| \}\\
     &\implica \\
     WP(Sc, 0 \leq i \leq |ciudades| \yLuego res &= \sum\limits_{j=0}^{i - 1} ciudades[j].habitantes)\\
     &\iff\\
     
\end{align*}
\indent
En conclusión, por monotonía, al haber probado (1), (2) y (3) queda demostrada la correctitud del programa según la lógica de la tripla de Hoare.

\end{document}
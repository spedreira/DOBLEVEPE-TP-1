\documentclass[10pt,a4paper,fleqn]{article}

\usepackage[spanish,activeacute,es-tabla]{babel}
\usepackage[utf8]{inputenc}
\usepackage{ifthen}
\usepackage{listings}
\usepackage{dsfont}
\usepackage{subcaption}
\usepackage{amsmath}
\usepackage[strict]{changepage}
\usepackage[top=1cm,bottom=2cm,left=1cm,right=1cm]{geometry}%
\usepackage{color}%
\newcommand{\tocarEspacios}{%
	\addtolength{\leftskip}{3em}%
	\setlength{\parindent}{0em}%
}

% Especificacion de procs

\newcommand{\In}{\textsf{in }}
\newcommand{\Out}{\textsf{out }}
\newcommand{\Inout}{\textsf{inout }}

\newcommand{\encabezadoDeProc}[4]{%
	% Ponemos la palabrita problema en tt
	%  \noindent%
	{\normalfont\bfseries\ttfamily proc}%
	% Ponemos el nombre del problema
	\ %
	{\normalfont\ttfamily #2}%
	\
	% Ponemos los parametros
	(#3)%
	\ifthenelse{\equal{#4}{}}{}{%
		% Por ultimo, va el tipo del resultado
		\ : #4}
}

\newenvironment{proc}[4][res]{%
	
	% El parametro 1 (opcional) es el nombre del resultado
	% El parametro 2 es el nombre del problema
	% El parametro 3 son los parametros
	% El parametro 4 es el tipo del resultado
	% Preambulo del ambiente problema
	% Tenemos que definir los comandos requiere, asegura, modifica y aux
	\newcommand{\requiere}[2][]{%
		{\normalfont\bfseries\ttfamily requiere}%
		\ifthenelse{\equal{##1}{}}{}{\ {\normalfont\ttfamily ##1} :}\ %
		\{\ensuremath{##2}\}%
		{\normalfont\bfseries\,\par}%
	}
	\newcommand{\asegura}[2][]{%
		{\normalfont\bfseries\ttfamily asegura}%
		\ifthenelse{\equal{##1}{}}{}{\ {\normalfont\ttfamily ##1} :}\
		\{\ensuremath{##2}\}%
		{\normalfont\bfseries\,\par}%
	}
	\renewcommand{\aux}[4]{%
		{\normalfont\bfseries\ttfamily aux\ }%
		{\normalfont\ttfamily ##1}%
		\ifthenelse{\equal{##2}{}}{}{\ (##2)}\ : ##3\, = \ensuremath{##4}%
		{\normalfont\bfseries\,;\par}%
	}
	\renewcommand{\pred}[3]{%
		{\normalfont\bfseries\ttfamily pred }%
		{\normalfont\ttfamily ##1}%
		\ifthenelse{\equal{##2}{}}{}{\ (##2) }%
		\{%
		\begin{adjustwidth}{+5em}{}
			\ensuremath{##3}
		\end{adjustwidth}
		\}%
		{\normalfont\bfseries\,\par}%
	}
	
	\newcommand{\res}{#1}
	\vspace{1ex}
	\noindent
	\encabezadoDeProc{#1}{#2}{#3}{#4}
	% Abrimos la llave
	\par%
	\tocarEspacios
}
{
	% Cerramos la llave
	\vspace{1ex}
}

\newcommand{\aux}[4]{%
	{\normalfont\bfseries\ttfamily\noindent aux\ }%
	{\normalfont\ttfamily #1}%
	\ifthenelse{\equal{#2}{}}{}{\ (#2)}\ : #3\, = \ensuremath{#4}%
	{\normalfont\bfseries\,;\par}%
}

\newcommand{\pred}[3]{%
	{\normalfont\bfseries\ttfamily\noindent pred }%
	{\normalfont\ttfamily #1}%
	\ifthenelse{\equal{#2}{}}{}{\ (#2) }%
	\{%
	\begin{adjustwidth}{+2em}{}
		\ensuremath{#3}
	\end{adjustwidth}
	\}%
	{\normalfont\bfseries\,\par}%
}

% Tipos

\newcommand{\nat}{\ensuremath{\mathds{N}}}
\newcommand{\ent}{\ensuremath{\mathds{Z}}}
\newcommand{\float}{\ensuremath{\mathds{R}}}
\newcommand{\bool}{\ensuremath{\mathsf{Bool}}}
\newcommand{\cha}{\ensuremath{\mathsf{Char}}}
\newcommand{\str}{\ensuremath{\mathsf{String}}}

% Logica

\newcommand{\True}{\ensuremath{\mathrm{true}}}
\newcommand{\False}{\ensuremath{\mathrm{false}}}
\newcommand{\Then}{\ensuremath{\rightarrow}}
\newcommand{\Iff}{\ensuremath{\leftrightarrow}}
\newcommand{\implica}{\ensuremath{\longrightarrow}}
\newcommand{\IfThenElse}[3]{\ensuremath{\mathsf{if}\ #1\ \mathsf{then}\ #2\ \mathsf{else}\ #3\ \mathsf{fi}}}
\newcommand{\yLuego}{\land _L}
\newcommand{\oLuego}{\lor _L}
\newcommand{\implicaLuego}{\implica _L}

\newcommand{\cuantificador}[5]{%
	\ensuremath{(#2 #3: #4)\ (%
		\ifthenelse{\equal{#1}{unalinea}}{
			#5
		}{
			$ % exiting math mode
			\begin{adjustwidth}{+2em}{}
				$#5$%
			\end{adjustwidth}%
			$ % entering math mode
		}
		)}
}

\newcommand{\existe}[4][]{%
	\cuantificador{#1}{\exists}{#2}{#3}{#4}
}
\newcommand{\paraTodo}[4][]{%
	\cuantificador{#1}{\forall}{#2}{#3}{#4}
}

%listas

\newcommand{\TLista}[1]{\ensuremath{seq \langle #1\rangle}}
\newcommand{\lvacia}{\ensuremath{[\ ]}}
\newcommand{\lv}{\ensuremath{[\ ]}}
\newcommand{\longitud}[1]{\ensuremath{|#1|}}
\newcommand{\cons}[1]{\ensuremath{\mathsf{addFirst}}(#1)}
\newcommand{\indice}[1]{\ensuremath{\mathsf{indice}}(#1)}
\newcommand{\conc}[1]{\ensuremath{\mathsf{concat}}(#1)}
\newcommand{\cab}[1]{\ensuremath{\mathsf{head}}(#1)}
\newcommand{\cola}[1]{\ensuremath{\mathsf{tail}}(#1)}
\newcommand{\sub}[1]{\ensuremath{\mathsf{subseq}}(#1)}
\newcommand{\en}[1]{\ensuremath{\mathsf{en}}(#1)}
\newcommand{\cuenta}[2]{\mathsf{cuenta}\ensuremath{(#1, #2)}}
\newcommand{\suma}[1]{\mathsf{suma}(#1)}
\newcommand{\twodots}{\ensuremath{\mathrm{..}}}
\newcommand{\masmas}{\ensuremath{++}}
\newcommand{\matriz}[1]{\TLista{\TLista{#1}}}
\newcommand{\seqchar}{\TLista{\cha}}
\newcommand{\listavacia}{\lange \rangle}

\renewcommand{\lstlistingname}{Código}
\lstset{% general command to set parameter(s)
	language=Java,
	morekeywords={endif, endwhile, skip},
	basewidth={0.47em,0.40em},
	columns=fixed, fontadjust, resetmargins, xrightmargin=5pt, xleftmargin=15pt,
	flexiblecolumns=false, tabsize=4, breaklines, breakatwhitespace=false, extendedchars=true,
	numbers=left, numberstyle=\tiny, stepnumber=1, numbersep=9pt,
	frame=l, framesep=3pt,
	captionpos=b,
}

\usepackage{caratula}
\usepackage{xcolor}
\usepackage{amssymb}
\usepackage{mathtools}

\titulo{Trabajo práctico 1: Especificación y WP}
\subtitulo{Segundo cuatrimestre de 2024}

\fecha{\today}

\materia{Algoritmos y Estructuras de Datos}
\grupo{Grupo DOBLEVEPE}

\integrante{Adra, Sofía Victoria}{71/24}{sofiavictoriaadra@gmail.com}
\integrante{Bermudez, Lucas Oscar}{124/24}{lucas.bermudez135@gmail.com}
\integrante{Pedreira, Santiago}{306/23}{spedreira.business@gmail.com}
\integrante{Troncoso, Fabian Lihuel}{379/24}{lihueltroncoso@gmail.com}
% Pongan cuantos integrantes quieran

% Declaramos donde van a estar las figuras
% No es obligatorio, pero suele ser comodo
\graphicspath{{../static/}}
\begin{document}

\maketitle

\section{Especificación}
Predicados comunes:
\begin{itemize}
    \item \pred{habitantesPositivos}{ciudades : \TLista{Ciudad}}{\paraTodo[unalinea]{c}{Ciudad}{c \in ciudades \implica c.habitantes \geq 0}}

    \item \pred{noHayRepetidos}{ciudades : \TLista{Ciudad}}{\paraTodo[unalinea]{i,j}{\ent}{(0 \leq i,j < |ciudades| \land i \neq j) \implicaLuego ciudades[i].habitantes \neq ciudades[j].habitantes}}

    \item \pred{esMatrizNoVacia}{m : \TLista{\TLista{\ent}}}{\paraTodo[unalinea]{a,b}{\TLista{\ent}}{(a \neq b \land a \in m \land b \in m ) \implica (\longitud a = \longitud b \land \longitud a > 0)}}

    \item \pred{esMatrizCuadrada}{m : \TLista{\TLista{\ent}}}{esMatrizNoVacia(m) \land \paraTodo[unalinea]{s}{\TLista{\ent}}{s \in m \implica (\longitud s = \longitud m)}}

    \item \pred{esSimétrica}{m : \TLista{\TLista{\ent}}}{esMatrizCuadrada(m) \yLuego \paraTodo[unalinea]{i,j}{\ent}{0 \leq i,j < \longitud m \implicaLuego (m[i][j] = m[j][i])}}

    \item \pred{diagonalDeCeros}{m : \TLista{\TLista{\ent}}}{esMatrizCuadrada(m) \yLuego \paraTodo[unalinea]{i}{\ent}{0 \leq i < \longitud m \implicaLuego (m[i][i] = 0)}}

    \item \pred {esUnCamino}{s : \TLista{\ent \times \ent}, distancias : \TLista{\TLista{\ent}}, desde : \ent, hasta : \ent}{(\longitud s > 0) \yLuego (s[0]_0 = desde \land s[\longitud s -1]_1 = hasta) \land \paraTodo[unalinea]{i}{\ent}{0 \leq i < \longitud s -1 \implicaLuego (s[i]_1 = s[i+1]_0)} \land \\ \paraTodo[unalinea]{i}{\ent}{0 \leq i < \longitud s \implicaLuego hayCaminoDirecto(distancias, s[i]_0, s[i]_1)}}
\end{itemize}


\subsection{grandesCiudades}
\begin{proc}{grandesCiudades}{\In ciudades : \TLista{Ciudad}}{\TLista{Ciudad}}
    \requiere{habitantesPositivos(ciudades) \land noHayRepetidos(ciudades)}
    \asegura{\paraTodo[unalinea]{x}{Ciudad}{x \in res \iff x \in ciudades \land x.habitantes > 50000 }}
\end{proc}


\subsection{sumaDeHabitantes}
\begin{proc}{sumaDeHabitantes}{\In menoresDeCiudades : \TLista{Ciudad}, \In mayoresDeCiudades : \TLista{Ciudad}}{\TLista{Ciudad}}
    \requiere{habitantesPositivos(menoresDeCiudades) \land noHayRepetidos(menoresDeCiudades)}
    \requiere{habitantesPositivos(mayoresDeCiudades) \land noHayRepetidos(mayoresDeCiudades)}
    \requiere{|menoresDeCiudades| = |mayoresDeCiudades| }
    \requiere{tienenMismosNombres(menoresDeCiudades, mayoresDeCiudades)}
    \asegura{|res| = |menoresDeCiudades|}
    \asegura{ tienenMismosNombres(res, menoresDeCiudades) }
    \asegura{\paraTodo[unalinea]{x}{Ciudad}{x \in res \implica \existe[unalinea]{i,j}{Ciudad}{i \in menoresDeCiudades \land  j \in mayoresDeCiudades} \land  x.nombre = i.nombre = j.nombre \land x.habitantes = i.habitantes + j.habitantes}}
\end{proc}

\pred{tienenMismosNombres}{ciudades1,ciudades2 : \TLista{Ciudad}}{\paraTodo[unalinea]{c}{Ciudad}{esCiudad(c,ciudades1) \iff esCiudad(c,ciudades2)}}

\pred{esCiudad}{c : Ciudad, ciudades : \TLista{Ciudad}}{\existe[unalinea]{e}{Ciudad}{e \in ciudades \land e.nombre = c.nombre}}



\subsection{hayCamino}

\begin{proc}{hayCamino}{\In distancias : \TLista{\TLista{\ent}}, \In desde : \ent, \In hasta : \ent}{Bool}
    \requiere{esMatrizCuadrada(distancias)\land esSimétrica(distancias) \land diagonalDeCeros(distancias)}
    \asegura{res = True \iff \existe[unalinea]{s}{\TLista{\ent \times \ent}}{esUnCamino(s,distancias,desde,hasta)}}
\end{proc}
%Arreglar la longitud de distancias
    \pred{hayCaminoDirecto}{distancias : \TLista{\TLista{\ent}}, desde : \ent, hasta : \ent}{(0 \leq desde,hasta < \longitud {distancias}) \yLuego (distancias[desde][hasta] \neq 0)}
        
\subsection{cantidadCaminosNSaltos}
\begin{proc}{cantidadCaminosNSaltos}{\Inout conexión : \TLista{\TLista{\ent}}, \In n : \ent}{}
	\requiere {n > 0 \land conexión = $ conexión_{0}$}
	\requiere {esMatrizCuadrada(conexión) \yLuego esDeOrdenUno(conexión)}
	\asegura {\existe[unalinea]{s}{\TLista{\TLista{\TLista{\ent}}}}{|s| = n \yLuego matricesEnTamaño(s, conexion_0) \\ \yLuego sonMatricesHastaOrdenN(s, n, $conexión_{0}$) \yLuego conexión = s[|s| - 1]}}
\end{proc}

\pred{esDeOrdenUno}{m : \TLista{\TLista{\ent}}}{\paraTodo[unalinea]{i,j}{\ent}{0 \leq i,j < |m| \implicaLuego (m[i][j] = 0 \lor m[i][j] = 1)}}
\pred{matricesEnTamaño}{s : \TLista{\TLista{\TLista{\ent}}}, c : \TLista{\TLista{\ent}}}{\paraTodo[unalinea]{m}{\TLista{\TLista{\ent}}}{m \in s \implicaLuego esMatrizCuadrada(m) \land |m| = |c|}}
\pred{sonMatricesHastaOrdenN}{s : \TLista{\TLista{\TLista{\ent}}}, n : \ent, m : \TLista{\TLista{\ent}}}{\paraTodo[unalinea]{i}{\ent}{0 < i < |s| \implicaLuego esProductoDeMatrices(s[i], s[i-1], m)}}
\pred{esProductoDeMatrices}{s,m',m : \TLista{\TLista{\ent}}}{\paraTodo[unalinea]{i,j}{\ent}{0 \leq i,j < |s| \implicaLuego s[i][j] = \sum\limits_{k=0}^{|s| - 1} m'[i][k] * m[k][j]}}

\subsection{caminoMinimo}

\begin{proc}{caminoMinimo}{\In origen : \ent, \In destino: \ent, \In distancias : \TLista{\TLista{\ent}}}{\TLista{\ent}}
	
	%    Requiere del ejercicio
	\requiere{esMatrizCuadrada(distancias) \land esSimetrica(distancias) \land todosNoNegativos(distancias) \\ \land diagonalDeCeros(distancias)}
	
	% Asegura de mi ejercicio
	\asegura{|res| > 0 \iff (\existe[unalinea]{s}{\TLista{\ent \times \ent}}{esElCaminoMinimo(s, distancias, origen, destino) \land (res[0] = s[0] _0) \land (res[|res| - 1] = s[|s|-1] _1) \land \paraTodo[unalinea]{i}{\ent}{0 \leq i < |s| - 1 \implicaLuego res[i+1] = s[i] _1}})}
	
\end{proc}
% Funcion auxiliar sumaDistancias
%Arreglar suma
\aux{sumaDistancias}{s: \TLista{\ent \times \ent}, distancias: \TLista{\TLista{\ent}}}{\ent}{$\sum_{i = 0}^{|s|-1}$ distancias[s[i]_0][s[i]_1]}

% Predicado esElCaminoMinimo
\pred{esElCaminoMinimo}{s: \TLista{\ent \times \ent}, distancias: \TLista{\TLista{\ent}, origen: \ent, destino: \ent}}{(esUnCamino(s, distancias, origen, destino)) \land \paraTodo[unalinea]{i}{\TLista{\ent \times \ent}}{esUnCamino(i, distancias, origen, destino) \implicaLuego sumaDistancias(s, distancias) \leq sumaDistancias(i, distancias)}}

\pred{todosNoNegativos}{m: \TLista{\TLista{\ent}}}{\paraTodo[unalinea]{s}{\TLista{\ent}}{s \in m \implica (\paraTodo[unalinea]{n}{\ent}{n \in s \implica n \geq 0)}}}

\section{Demostraciones de correctitud}
\subsection{}
Para demostrar que la implementación es correcta respecto con respecto de la especificación, debemos probar la validez de la tripla de Hoare  \{P\}S\{Q\}, donde P es la precondición del programa, S el programa y Q la postcondición. La tripla de Hoare es válida sii:\\
$\{P\}S\{Q\} \iff P \implica wp(S,Q)$\\
Para eso probaremos:
\begin{enumerate}
    \item P \implica wp($S_1;S_2, P_c$)
    \item[] Donde $S_1;S_2$ son las instrucciones previas al inicio del ciclo, y $P_c$ es la precondición del mismo.
    \item $P_c$ \implica wp(while...,$Q_c$)
    \item[] Donde $Q_c$ es la postcondición del ciclo.
    \item $Q_c$ \implica wp($S_3$,Q)
    \item[] Donde $S_3$ es la instrucción posterior al ciclo. Como en este caso, el programa termina con el fin del ciclo, probaremos $Q_c$ \implica Q.
\end{enumerate}
Por monotonía, esto nos permite demostrar que P \implica wp(S,Q) es verdadera. Definimos precondición, postcondición del ciclo, invariante y función variante para la demostración.\\
\begin{eqnarray}
    Q_c &\equiv& \{res = \small \displaystyle \sum_{j=0}^{|ciudades|-1} ciudades[j].habitantes \ \land \ i = |ciudades|\} \notag \\
    P_c &\equiv& \{res=0 \land i=0 \} \notag \\
    I &\equiv& \{ 0 \leq i \leq |ciudades| \yLuego (res = \small \displaystyle \sum_{j=0}^{i-1} ciudades[j].habitantes)\} \notag \\
    Q &\equiv& \{res = \sum\limits_{j=0}^{|ciudades| - 1} ciudades[j].habitantes\} \notag \\
    P &\equiv& \{ \existe[unalinea]{j}{\ent}{0 \leq j < |ciudades| \yLuego ciudades[j].habitantes > 50.000} \land \notag \\
    & & \paraTodo[unalinea]{j}{\ent}{0 \leq j < |ciudades| \implicaLuego ciudades[j].habitantes \geq 0}
    \ \land \notag \\ & & \paraTodo[unalinea]{k, h}{\ent}{0 \leq k < h < |ciudades| \implicaLuego ciudades[k].nombre \neq ciudades[h].nombre} \} \notag \\
    B &\equiv& \{ i < |ciudades| \} \notag \\
    fv &=& |ciudades| - i \notag
\end{eqnarray}
\item 1) Comenzamos probando P \implica $wp(S_1;S_2, P_c$)\\
\begin{eqnarray}
    wp(S_1;S_2, P_c) &\overset{Ax3}{\equiv}& wp(S_1; wp(S_2,\ P_c))  \notag \\
    &\equiv& wp(res:= 0, wp(i:=0,(res = 0 \land i = 0)) \notag \\
    &\overset{Ax1}{\equiv}& wp(res:= 0,def(0) \yLuego P_c_{0}^{i}) \notag \\
    &\equiv& wp(res:= 0, True \ \yLuego \ res + 0 \land 0 =0) \notag \\
    &\equiv& wp(res:=0, res = 0 \land True) \notag \\
    \item q = \{res = 0\} \notag \\
    &\overset{Ax1}{\equiv}& def(0) \yLuego q_{0}^{res} \notag \\
    &\equiv& True \yLuego (0 = 0) \notag \\
    &\equiv& True \land True \notag \\
    wp(S_1;S_2, P_c) &\equiv& True \notag 
\end{eqnarray}
Si vemos la implicación: 
$ P \implica wp(S_1;S_2, P_c) \equiv P \implica True \equiv True$\\

 \item 3) Continuamos probando $Q_c$ \implica Q\\
 Es clara la implicancia pues la postocndición del ciclo y del programa son idénticas.\\
 
 \item 2) Finalmente, para probar $P_c$ \implica wp(while...,$Q_c$), se debe demostrar la correctitud del ciclo mediante Teorema del Invariante, y la terminación del mismo mediante el Teorema de Terminación.\\
 Teorema del Invariante \\
 \begin{itemize}
    \item[a.] $P_c \implica I$ 
    \item[b.] $\{I \land B\}S_c\{I\}$
    \item[c.]  $I \land \neg B \implica Q_c$
\end{itemize}
\item a. Como queremos analizar la implicación, podemos asumir el antecedente verdadero.\\
\{$res=0 \ \land \ i=0 \} \implica  \{0 \leq i \leq |ciudades| \yLuego (res = \small \displaystyle \sum_{j=0}^{i-1} ciudades[j].habitantes)$ \}

$P_c \implica 0 \leq \textcolor{blue}{0} \leq |ciudades| \ \yLuego \ (\textcolor{blue}{0} = \small \displaystyle \sum_{j=0}^{\textcolor{blue}{0}-1} ciudades[j].habitantes) \equiv 0 \leq 0 \leq |ciudades| \yLuego (0 = \displaystyle \sum_{j=0}^{-1} ciudades[j].habitantes)$\\
Como $0 \leq 0 \equiv True$, y la sumatoria de un rango vacío es 0: \\
\equiv \ $P_c \implica 0 \leq |ciudades| \yLuego (0=0)$ \\
Como la longitud de una secuencia siempre es 0 o mayor:\\
\equiv \ $P_c \implica True \land True \equiv \ $P_c \implica True \equiv \  True$ \\

\item b. La tripla de Hoare, $\{I \land B\}S_c\{I\}$, es válida si y sólo si $I \land B\ \implica wp(S_c,I)$.El cuerpo del ciclo tiene dos instrucciones, que llamaremos $S_c1$ y $S_c2$.         
\begin{eqnarray}
    wp(S_c1;S_c2, P_c) &\overset{Ax3}{\equiv}& wp(S_c1; wp(S_c2,\ P_c))  \notag \\
    &\equiv& wp(res:= res + ciudades[i].habitantes, wp(i:= i + 1,I) \notag
    \end{eqnarray}
   Calculamos la wp anidada $W_2 = wp(i:= i + 1,I)$
\begin{eqnarray}
    wp(i:= i + 1,I) &\overset{Ax1}{\equiv}& wp(res:= 0,def(0) \yLuego P_c_{0}^{i}) \notag \\
    &\equiv& def(i+1) \yLuego I_{i+1}^{i}) \notag \\
    &\equiv& (def(i) \land def(1)) \yLuego 0 \leq \textcolor{blue}{i+1} \leq |ciudades| \yLuego \notag \\
    & & (res = \small \displaystyle \sum_{j=0}^{(\textcolor{blue}{i+1})-1} ciudades[j].habitantes)) \notag \\
    W_2 &\equiv& -1 \leq i \leq |ciudades| \yLuego (res = \small \displaystyle \sum_{j=0}^{i} ciudades[j].habitantes) \notag
\end{eqnarray}
Ahora podemos calcular la wp principal: 
\begin{eqnarray} 
    wp($S_c1$,W_2) &\overset{Ax1}{\equiv}& (def(res+ ciudades[i].habitantes) \yLuego W_2_{res+ ciudades[i].habitantes}^{res} \notag \\
    &\equiv& (def(res) \land def(ciudades) \land def(i)) \yLuego 0 \leq i < |ciudades|  \yLuego \notag \\
    & & (-1 \leq i \leq |ciudades| \yLuego (\textcolor{blue}{res+ ciudades[i].habitantes} = \small \displaystyle \sum_{j=0}^{i} ciudades[j].habitantes) \notag \\
    &\equiv& (0 \leq i < |ciudades| \yLuego -1 \leq i \leq |ciudades| \yLuego res = \small \displaystyle \sum_{j=0}^{i} ciudades[j].habitantes - ciudades[i].habitantes) \notag
\end{eqnarray}
Como ciudades[i].habitantes es el último término de la sumatoria, podemos reescribirla. Además para el rango de i queda el más fuerte, es decir:  
\begin{eqnarray} 
 wp($S_c1$,W_2) &\equiv& (0 \leq i < |ciudades| \yLuego res = \small \displaystyle \sum_{j=0}^{i-1} ciudades[j].habitantes) \notag 
\end{eqnarray}
Ahora retomando la implicación, si asumimos el antecedente $I \land B$ cierto: \\
\begin{itemize}
    \item \parbox[t]{\textwidth}{$I \land \ B \implica 0 \leq i \leq |ciudades| \land i < |ciudades| \implica 0 \leq i < |ciudades|$}
    \item \parbox[t]{\textwidth}{La sumatoria está idéntica en I y WP.}
\end{itemize} 
Como probamos $I \land B\ \implica wp(S_c,I)$, la tripla de Hoare es válida.\\

\item c. Como queremos analizar la implicación $I \land \neg B \implica Q_c$, podemos asumir el antecedente verdadero. 
\begin{eqnarray}
    I \land \neg B &\equiv& 0 \leq i \leq |ciudades| \yLuego (res = \small \displaystyle \sum_{j=0}^{i-1} ciudades[j].habitantes) \land \neg(i<|ciudades|) \\
    &\equiv& i=|ciudades| \yLuego (res = \small \displaystyle \sum_{j=0}^{i-1} ciudades[j].habitantes) \notag 
\end{eqnarray}
 (1) $0 \leq i \leq |ciudades| \land \neg(i<|ciudades|)\equiv 0 \leq i \leq |ciudades| \land i\geq |ciudades| \implica i = |ciudades|$\\
 
 Veamos la implicación, reemplazando $i = |ciudades|$:\\
$ I \land B \implica i = |ciudades| \implica res = \small \displaystyle \sum_{j=0}^{\textcolor{blue}{|ciudades|}-1} ciudades[j].habitantes$\\
Así, $I \land \neg B \implica Q_c$ es cierto\\ 
Al haberse probado (a),(b) y (c) queda demostrado que el ciclo es parcialmente correcto respecto la especificación. Ahora, se demuestra la finalización del ciclo.\\
\newpage
Teorema de Terminación
\begin{itemize}
    \item[d.] $\{I \land B \land fv = v_0\}S_c \{fv<v_0\}$ 
    \item[e.] $I \land fv \leq 0 \implica \neg B$
\end{itemize}
\indent
La proposición (d), se prueba de la siguiente manera: \\
\small \text{Definimos F =  \{fv $<$ $V_{0}$\}}  \\
\small \text{La tripla de Hoare es verdadera si y sólo si:} 
\begin{align*}
    &\{I \land B \land fv = V_{0} \}\implica WP(Sc, F) 
\end{align*}
\text{Se calcula la WP como}  
\begin{align*}
	W_{1} &\equiv WP(res := res + ciudades[i].habitantes ; i := i+1 , F)  \\ 
    &\overset{Ax3}{\equiv} WP(res := res + ciudades[i].habitantes, \textcolor{blue}{WP(i := i+1, F)}) 
\end{align*}
\text{Se calcula el WP anidado}
\begin{align*}
	& \textcolor{blue}{WP(i := i+1, F)} \overset{Ax1}{\equiv} \{ def(i+1) \yLuego F_{i+1}^{i} \} \equiv \{ true \yLuego |ciudades| - i - 1 < V_{0} \}  \\
    &\equiv \{\overbrace{\textcolor{blue}{|ciudades| - i - 1 < V_{0}}}^{W_2}\} 
\end{align*}
\text{Reemplazamos en la original}
\begin{align*}
    W_{1} &\equiv WP(res := res + ciudades[i].habitantes, \textcolor{blue}{W_2})  \\
    &\overset{Ax1}{\equiv} WP(def(res + ciudades[i].habitantes) \yLuego W_{2 res + ciudades[i].habitantes}^{res})  \\
    &\equiv  \{ true \yLuego |ciudades| - i - 1 < V_{0} \} \equiv W_1 
\end{align*}
&\text{Retomando la implicación de la tripla de Hoare de la proposición (d), ya se puede reemplazar la WP por $W_{1}$} 
\begin{align*}
	&\{I \land B \land fv = V_{0} \} \implica \{ |ciudades| - i - 1 < V_{0} \}  \\
    &\text{Asumimos el antecedente verdadero, por lo que solo dependemos del consecuente}  \\
	& \{ |ciudades| - i - 1 < fv \} \underset{V_0 = fv}{\equiv} \{ |ciudades| - i - 1 < |ciudades| - i \} \equiv \{ - 1 < 0 \}  \\
	&\text{Luego}  \\
	&\{I \land B \land fv = V_{0} \} \implica \{ true \}  \\
	&\text{Que es tautología} 
\end{align*}
\indent

\noindent La proposición (e), se prueba de la siguiente manera:
\begin{align*}
    &\{ I \land |ciudades| - i \leq 0 \} \implica i \geq |ciudades|  \\
    &{\parbox{\textwidth}{ {Si operamos un poco}}}  \\
    &|ciudades| - i \leq 0 \equiv i \geq |ciudades|  \\
    &{\parbox{\textwidth}{ {Luego}}}  \\
	&\{ I \land i \geq |ciudades| \} \implica i \geq |ciudades|  \\
    &\text{Asumimos el antecedente verdadero y nos queda}  \\
    &i \geq |ciudades| \underset{i \geq |ciudades|}{\equiv} True  \\
    &{\parbox{\textwidth}{ {Que es tautología}}}  
\end{align*}


\noindent \text{Al haberse probado (d) y (e) queda demostrado que el ciclo finaliza por el teorema de finalización, y por lo tanto,} \\ 
\noindent \text{como ya se probó el teorema del invariante, queda probado (2).}\\

\noindent En conclusión, por monotonía, al haber probado (1), (2) y (3) queda demostrada la correctitud del programa según la lógica de la tripla de Hoare.
\subsection{}
Para demostrar que el valor devuelto es mayor a 50.000, definimos una nueva postcondición del programa. 
\vspace{2,5mm}\\
Q \equiv \{ res = $\small \displaystyle \sum_{j=0}^{|ciudades|-1} ciudades[j].habitantes \land res >$ 50.000 \} \equiv \{$\small \displaystyle \sum_{j=0}^{|ciudades|-1} ciudades[j].habitantes >$ 50.000 \}\\
Debemos probar la validez de la tripla de Hoare \{P\}S\{Q\}, para eso probaremos:
\begin{enumerate}
    \item P \implica wp($S_1;S_2, P_c$)
    \item $P_c$ \implica wp(while...,$Q_c$)
    \item $Q_c$ \implica Q
\end{enumerate}
Por monotonía, esto nos permite demostrar que P \implica wp(S,Q) es verdadera. Definimos una nueva precondición y postcondición del ciclo, e invariante.\\
\begin{eqnarray}
    Q_c &\equiv& \{\small \displaystyle \sum_{j=0}^{|ciudades|-1} ciudades[j].habitantes > 50.000\} \notag \\
    P_c &\equiv& \{ \existe[unalinea]{k}{\ent}{0 \leq k< |ciudades| \yLuego ciudades[k].habitantes >  50.000 }\land \notag \\ 
    & & \paraTodo[unalinea]{j}{\ent}{0\leq j < |ciudades| \implicaLuego ciudades[j].habitantes \geq 0} \land res=0 \land i=0 \} \notag \\
    I &\equiv& \{ 0 \leq i \leq |ciudades| \yLuego (res = \small \displaystyle \sum_{j=0}^{i-1} ciudades[j].habitantes) \land \notag \\
    & & \existe[unalinea]{k}{\ent}{0 \leq k< |ciudades| \yLuego ciudades[k].habitantes >  50.000 }\land \paraTodo[unalinea]{j}{\ent}{0 \leq j < |ciudades| \implicaLuego ciudades[j].habitantes \geq 0} \} \notag 
\end{eqnarray}
\item 1) P \implica $wp(S_1;S_2, P_c$)\\ \vspace{-10mm}

\begin{eqnarray}
    wp(S_1;S_2, P_c) &\overset{Ax3}{\equiv}& wp(S_1; wp(S_2,\ P_c))  \notag \\
    &\equiv& wp(S_1, wp(i:=0,\ P_c)) \notag \\
    &\overset{Ax1}{\equiv}& wp(S_1, def(0) \yLuego \text{$P_c$}_{0}^{i}) \notag \\
    &\equiv& wp(S_1, True \yLuego \existe[unalinea]{k}{\ent}{0 \leq k< |ciudades| \yLuego ciudades[k].habitantes >  50.000 }\land \notag \\ & & \paraTodo[unalinea]{j}{\ent}{0\leq j < |ciudades| \implicaLuego ciudades[j].habitantes \geq 0}\ \land \ res=0\ \land \ \textcolor{blue}{0=0}) \notag \\
     &\equiv& wp(res:= 0, \existe[unalinea]{k}{\ent}{0 \leq k< |ciudades| \yLuego ciudades[k].habitantes >  50.000 }\land \notag \\ & & \paraTodo[unalinea]{j}{\ent}{0\leq j < |ciudades| \implicaLuego ciudades[j].habitantes \geq 0}\ \land \ res=0\ \land \ \textcolor{blue}{True}) \notag \\
     &\overset{Ax1}{\equiv}& \existe[unalinea]{k}{\ent}{0 \leq k< |ciudades| \yLuego ciudades[k].habitantes >  50.000 }\land \notag \\ & & \paraTodo[unalinea]{j}{\ent}{0\leq j < |ciudades| \implicaLuego ciudades[j].habitantes \geq 0}\ \land \ \textcolor{blue}{0=0}) \notag \\
       wp(S_1;S_2, P_c) &\equiv& \existe[unalinea]{k}{\ent}{0 \leq k< |ciudades| \yLuego ciudades[k].habitantes >  50.000 }\land \notag \\ & & \paraTodo[unalinea]{j}{\ent}{0\leq j < |ciudades| \implicaLuego ciudades[j].habitantes \geq 0} \notag 
\end{eqnarray}
Luego, P \implica \ $wp(S_1;S_2, P_c$) es cierto pues ambos cuantificadores están idénticos en la WP y P. 

\item 2) $P_c$\implica \ $wp(while...,Q_c$)\\
Debemos probar la correctitud y terminación del ciclo.
\item 1.\ $P_c\implica I$ \\

\begin{itemize}
    \item \parbox[t]{\textwidth}{ Ambos cuantificadores, \existe[unalinea]{k}{\ent}{0 \leq k< |ciudades| \yLuego ciudades[k].habitantes >  50.000 } y \\ \paraTodo[unalinea]{j}{\ent}{0\leq j < |ciudades| \implicaLuego ciudades[j].habitantes \geq 0}están idénticos en I y P_c} 
    \item \parbox[t]{\textwidth}{res = 0 \land \ i = 0 \implica res = $\small \displaystyle \sum_{j=0}^{i-1} ciudades[j].habitantes$ = $\small \displaystyle \sum_{j=0}^{-1} ciudades[j].habitantes$ = 0 por el rango vacío }
     \item \parbox[t]{\textwidth}{i = 0 \implica \ 0 \leq 0 \leq |ciudades|}
\end{itemize}
\item 2.\ $\{I \land B\}S_c\{I\}$  \\Para probar que la tripla de Hoare es válida, vemos que $\{I \land B\} \implica wp(S_c,I)$ sea cierto.
El cuerpo del ciclo tiene dos instrucciones, que llamaremos $S_c1$ y $S_c2$. 

\begin{eqnarray}
    wp(S_c1;S_c2, I) &\overset{Ax3}{\equiv}& wp(S_c1, wp(S_c2, I)) \notag \\
     &\equiv&  wp(S_c1, wp(i:= i+1, I))\notag \\
      &\overset{Ax1}{\equiv}& wp(S_c1,def(i+1) \yLuego I_{i}^{i+1})\notag \\
      &\equiv&  wp(S_c1, (def(i) \land def(1)) \yLuego 0 \leq \textcolor{blue}{i+1} \leq |ciudades| \yLuego (res = \small \displaystyle \sum_{j=0}^{\textcolor{blue}{i+1}-1} ciudades[j].habitantes \land \notag \\ & & \existe[unalinea]{k}{\ent}{0 \leq k< |ciudades| \yLuego ciudades[k].habitantes >  50.000 }\land \notag \\ & & \paraTodo[unalinea]{j}{\ent}{0\leq j < |ciudades| \implicaLuego ciudades[j].habitantes \geq 0})) \notag \\
       &\equiv& wp(res:= res + ciudades[i].habitantes, (True \land True) \yLuego \ 0 \leq i+1 \leq |ciudades| \ yLuego \notag \\ & & (res = \small \displaystyle \sum_{j=0}^{i} ciudades[j].habitantes \land \notag \\ & & \existe[unalinea]{k}{\ent}{0 \leq k< |ciudades| \yLuego ciudades[k].habitantes >  50.000 }\land \notag \\ & & \paraTodo[unalinea]{j}{\ent}{0\leq j < |ciudades| \implicaLuego ciudades[j].habitantes \geq 0})) \notag \\
       &\overset{Ax1}{\equiv}& ((def(res) \land def(ciudades) \land def(i))\yLuego 0 \leq i < |ciudades|) \yLuego (0 \leq i+1 \leq |ciudades|) \ \land \notag  \\ & & \textcolor{blue}{res + ciudades[i].habitantes} = \small \displaystyle \sum_{j=0}^{i} ciudades[j].habitantes \land \\ & & \existe[unalinea]{k}{\ent}{0 \leq k< |ciudades| \yLuego ciudades[k].habitantes >  50.000 }\land \notag \\ & & \paraTodo[unalinea]{j}{\ent}{0\leq j < |ciudades| \implicaLuego ciudades[j].habitantes \geq 0} \notag \\
        &\equiv& (0 \leq i < |ciudades| \ \land \ 0 \leq i+1 \leq |ciudades|) \ \yLuego (res= \small \displaystyle \sum_{j=0}^{i-1} ciudades[j].habitantes \land \\ & & \existe[unalinea]{k}{\ent}{0 \leq k< |ciudades| \yLuego ciudades[k].habitantes >  50.000 }\land \notag \\ & & \paraTodo[unalinea]{j}{\ent}{0\leq j < |ciudades| \implicaLuego ciudades[j].habitantes \geq 0}) \notag \\
        wp(S_c1;S_c2, I)  &\equiv& (0 \leq i <|ciudades|) \yLuego (res= \small \displaystyle \sum_{j=0}^{i-1} ciudades[j].habitantes \land \notag \\ & & \existe[unalinea]{k}{\ent}{0 \leq k< |ciudades| \yLuego ciudades[k].habitantes >  50.000 }\land \notag \\ & & \paraTodo[unalinea]{j}{\ent}{0\leq j < |ciudades| \implicaLuego ciudades[j].habitantes \geq 0}) \notag 
\end{eqnarray}
(2) Si despejamos: $res = \small \displaystyle \sum_{j=0}^{i} ciudades[j].habitantes \ - ciudades[i].habitantes = \small \displaystyle \sum_{j=0}^{i-1} ciudades[j].habitantes$\\
(3) $(0 \leq i < |ciudades| \ \land \ 0 \leq i+1 \leq |ciudades|)\equiv(0 \leq i < |ciudades| \ \land \ -1 \leq i \leq |ciudades|-1)\\
\implica 0 \leq i <|ciudades|$\\
Ahora vemos la implicación: \\
\begin{itemize}
    \item \parbox[t]{\textwidth}{$I \land \ B \implica 0 \leq i \leq |ciudades| \land i < |ciudades| \implica 0 \leq i < |ciudades|$}
    \item \parbox[t]{\textwidth}{Tanto la sumatoria como los cuantificadores están idénticos en I y WP.}
\end{itemize}

\item 3. $I \land \neg B \implica Q_c$\\
\begin{eqnarray}
    I \land \neg B &\equiv& 0 \leq i \leq |ciudades| \yLuego (res = \small \displaystyle \sum_{j=0}^{i-1} ciudades[j].habitantes \land \notag \\ & &\existe[unalinea]{k}{\ent}{0 \leq k< |ciudades| \yLuego ciudades[k].habitantes >  50.000 }\land \notag \\ & & \paraTodo[unalinea]{j}{\ent}{0\leq j < |ciudades| \implicaLuego ciudades[j].habitantes \geq 0}) \land \neg(i<|ciudades|) \\
    &\equiv& i=|ciudades| \yLuego (res = \small \displaystyle \sum_{j=0}^{i-1} ciudades[j].habitantes \land \notag \\ & &\existe[unalinea]{k}{\ent}{0 \leq k< |ciudades| \yLuego ciudades[k].habitantes >  50.000 }\land \notag \\ & & \paraTodo[unalinea]{j}{\ent}{0\leq j < |ciudades| \implicaLuego ciudades[j].habitantes \geq 0}) \notag 
 \end{eqnarray}
 (4) $0 \leq i \leq |ciudades| \land \neg(i<|ciudades|)\equiv 0 \leq i \leq |ciudades| \land i\geq |ciudades| \implica i = |ciudades|$\\
 Veamos la implicación:
\begin{itemize}
    \item \parbox[t]{\textwidth}{Como asumimos el antecedente $I \land \neg B$ cierto, podemos reemplazar en la sumatoria.\\ res = $\small \displaystyle \sum_{j=0}^{\textcolor{blue}{|ciudades|}-1} ciudades[j].habitantes$\\
    Además, si reescribimos la sumatoria: \\
    res = $\small \displaystyle \sum_{j=0}^{k-1} ciudades[j].habitantes \ + ciudades[k].habitantes + \small \displaystyle \sum_{j=k+1}^{|ciudades|-1} ciudades[j].habitantes$\\
    De esta forma, sabemos por asumir el antecedente verdadero:\\
    \paraTodo[unalinea]{j}{\ent}{0\leq j < |ciudades| \implicaLuego (ciudades[j].habitantes \geq 0 \land \\ \existe[unalinea]{k}{\ent}{0 \leq k< |ciudades| \yLuego (ciudades[k].habitantes >  50.000 \land \\res = $\small \displaystyle \sum_{j=0}^{k-1} ciudades[j].habitantes \ + ciudades[k].habitantes + \small \displaystyle \sum_{j=k+1}^{|ciudades|-1} ciudades[j].habitantes$)})} \\\\ }
    Veamos que los limites de las sumatorias nunca se indefinen:
    \item \parbox[t]{\textwidth}{Como $k <|ciudades| \implica k-1<|ciudades|$ se ve que nunca se indefine ningun termino de la sumatoria. \\
    \paraTodo[unalinea]{j}{\ent}{0\leq j < |ciudades| \implicaLuego ciudades[j].habitantes \geq 0} \implica  \displaystyle \sum_{j=0}^{k-1} ciudades[j].habitantes \geq 0}
     \item \parbox[t]{\textwidth}{Como $0\leq k \implica 0\leq k+1, y \ k<|ciudades| \implica k+1\leq|ciudades|$, se ve que nunca se indefine ningun termino de \\ la sumatoria(a lo mucho el rango es vacío si $k+1= |ciudades|$). \\
      \paraTodo[unalinea]{j}{\ent}{0\leq j < |ciudades| \implicaLuego ciudades[j].habitantes \geq 0} \implica  \displaystyle \sum_{j=k+1}^{|ciudades|-1} ciudades[j].habitantes \geq 0}
       \item \parbox[t]{\textwidth}{Así, si vale $I \land \neg B$:\\
       \implica res \geq 0 +  ciudades[k].habitantes + 0 \\
       Y como \existe[unalinea]{k}{\ent}{0 \leq k< |ciudades| \yLuego (ciudades[k].habitantes >  50.000 \land \\res = $\small \displaystyle \sum_{j=0}^{k-1} ciudades[j].habitantes \ + ciudades[k].habitantes + \small \displaystyle \sum_{j=k+1}^{|ciudades|-1} ciudades[j].habitantes$)} \\
       \implica \ $res > 50.000 \equiv $\small \displaystyle \sum_{j=0}^{|ciudades|-1} ciudades[j].habitantes > 50.000$ }
\end{itemize}
Así, $I \land \neg B \implica Q_c$ es cierto\\
\item 4. $\{I \land B \land fv=v_0\}S_c\{fv<v_0\}$\\Para probar que la tripla de Hoare es válida, vemos que $(I \land B \land fv=v_0) \implica wp(S_c,fv<v_0)$ sea cierto.\\
\begin{eqnarray}
    wp(S_c1;S_c2, fv<v_0) &\overset{Ax3}{\equiv}& wp(S_c1, wp(S_c2, fv<v_0)) \notag \\
     &\equiv&  wp(S_c1, wp(i:= i+1, \textcolor{blue}{|ciudades|-i < v_0}))\notag \\
    &\overset{Ax1}{\equiv}& wp(S_c1, def(i+1) \yLuego (|ciudades|- (\textcolor{blue}{i+1}) < v_0)) \notag \\
    &\equiv& wp(res:= res + ciudades[i].habitantes, (def(i) \land def(1)) \yLuego (|ciudades| -i - 1 < v_0)\notag \\
    &\overset{Ax1}{\equiv}& (def(res) \land def(ciudades) \land def(i))\yLuego 0 \leq i < |ciudades| \yLuego (|ciudades| -i -1 < v_0) \notag \\
    wp(S_c1;S_c2, fv<v_0) &\equiv& 0 \leq i < |ciudades| \land |ciudades| -i -1 < v_0 \notag 
\end{eqnarray}
Ahora vemos la implicación: \\
\begin{itemize}
    \item \parbox[t]{\textwidth}{$(I \land B \land fv=v_0) \implica 0 \leq i \leq |ciudades| \ \land \ i < |ciudades|\ \implica 0 \leq i < |ciudades| $}
    \item \parbox[t]{\textwidth}{$(I \land B \land fv=v_0) \implica |ciudades|-1=v_0 \implica |ciudades| -i -1 < v_0 = |ciudades| -i -1 < \textcolor{blue}{|ciudades| -i} \Leftrightarrow -1<0$}
\end{itemize}
\item 5. $I \land fv\leq 0 \implica \neg B$
\begin{itemize}
    \item \parbox[t]{\textwidth}{I \land fv\leq 0 \implica 0\leq i \leq |ciudades| \land \textcolor{blue}{|ciudades|-i\leq 0} \equiv 0 \leq i \leq|ciudades| \land |ciudades| \leq i \implica i=|ciudades|\\ \implica \neg(i<|caminos|)\equiv\neg B}
\end{itemize}

Al probar la correctitud y terminación del ciclo, probamos $P_c$\implica \ $wp(while...,Q_c$).\\
\item 3) $Q_c \implica Q$\\
Es clara la implicancia pues la postcondición del ciclo y la del programa son idénticas.

\noindent En conclusión, como (1), (2), (3), (4) y (5) fueron probados con esta nueva postcondición queda a su vez demostrado que el valor del resultado es mayor a 50000 como queriamos ver.
\end{document}
